Thank you for considering our submission: Deep Predictive Learning in Neocortex and Pulvinar.

This paper presents a novel, computationally-explicit theory for the function of the thalamocortical circuits between the deep layers of the neocortex and the pulvinar nucleus of the thalamus.  Existing theories for these circuits include roles in modulation, attention, and synchronization across brain areas.  Our theory is that these circuits also support predictive error-driven learning, which is a widely-embraced principle, but the existing hypotheses about the neural mechanisms supporting this form of learning have not been strongly empirically supported.  By contrast, our theory builds directly on the basic, well-established features of these thalamocortical circuits.  Thus, this work provides a new way to think about a major neural system, and an important new direction for thinking about how learning overall might work more generally.

This work should have broad appeal across the spectrum of neuroscience, by virtue of the broad general interest in learning, and predictive learning more specifically.  Furthermore, we address detailed neuroanatomy, physiology, and experiments at multiple levels that have investigated the alpha frequency oscillations associated with the thalamocortical circuitry.  This work also has implications for computational learning theory and the implications of deep learning models for neuroscience.  We present an initial implementation of the theory that learns abstract categorical representations that qualitatively match those in the ventral visual stream.

Although there is much more work to be done, we think this paper represents a substantial first step toward developing an important new theory for a major aspect of brain function.  Thank you for your consideration.

- Randall O'Reilly on behalf of the co-Authors.

